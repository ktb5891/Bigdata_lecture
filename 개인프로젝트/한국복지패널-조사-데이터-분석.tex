% Options for packages loaded elsewhere
\PassOptionsToPackage{unicode}{hyperref}
\PassOptionsToPackage{hyphens}{url}
%
\documentclass[
]{article}
\usepackage{amsmath,amssymb}
\usepackage{lmodern}
\usepackage{ifxetex,ifluatex}
\ifnum 0\ifxetex 1\fi\ifluatex 1\fi=0 % if pdftex
  \usepackage[T1]{fontenc}
  \usepackage[utf8]{inputenc}
  \usepackage{textcomp} % provide euro and other symbols
\else % if luatex or xetex
  \usepackage{unicode-math}
  \defaultfontfeatures{Scale=MatchLowercase}
  \defaultfontfeatures[\rmfamily]{Ligatures=TeX,Scale=1}
\fi
% Use upquote if available, for straight quotes in verbatim environments
\IfFileExists{upquote.sty}{\usepackage{upquote}}{}
\IfFileExists{microtype.sty}{% use microtype if available
  \usepackage[]{microtype}
  \UseMicrotypeSet[protrusion]{basicmath} % disable protrusion for tt fonts
}{}
\makeatletter
\@ifundefined{KOMAClassName}{% if non-KOMA class
  \IfFileExists{parskip.sty}{%
    \usepackage{parskip}
  }{% else
    \setlength{\parindent}{0pt}
    \setlength{\parskip}{6pt plus 2pt minus 1pt}}
}{% if KOMA class
  \KOMAoptions{parskip=half}}
\makeatother
\usepackage{xcolor}
\IfFileExists{xurl.sty}{\usepackage{xurl}}{} % add URL line breaks if available
\IfFileExists{bookmark.sty}{\usepackage{bookmark}}{\usepackage{hyperref}}
\hypersetup{
  hidelinks,
  pdfcreator={LaTeX via pandoc}}
\urlstyle{same} % disable monospaced font for URLs
\usepackage[margin=1in]{geometry}
\usepackage{graphicx}
\makeatletter
\def\maxwidth{\ifdim\Gin@nat@width>\linewidth\linewidth\else\Gin@nat@width\fi}
\def\maxheight{\ifdim\Gin@nat@height>\textheight\textheight\else\Gin@nat@height\fi}
\makeatother
% Scale images if necessary, so that they will not overflow the page
% margins by default, and it is still possible to overwrite the defaults
% using explicit options in \includegraphics[width, height, ...]{}
\setkeys{Gin}{width=\maxwidth,height=\maxheight,keepaspectratio}
% Set default figure placement to htbp
\makeatletter
\def\fps@figure{htbp}
\makeatother
\setlength{\emergencystretch}{3em} % prevent overfull lines
\providecommand{\tightlist}{%
  \setlength{\itemsep}{0pt}\setlength{\parskip}{0pt}}
\setcounter{secnumdepth}{-\maxdimen} % remove section numbering
\ifluatex
  \usepackage{selnolig}  % disable illegal ligatures
\fi

\author{}
\date{\vspace{-2.5em}}

\begin{document}

getwd() setwd("C:/Users/ktb58/OneDrive/바탕
화면/210308/14\_R프로그래밍/개인프로젝트\_복지패널/(2020년 15차
한국복지패널조사) 데이터(beta1.1)\_spss")

\hypertarget{uxd328uxd0a4uxc9c0-uxc900uxbe44}{%
\section{패키지 준비}\label{uxd328uxd0a4uxc9c0-uxc900uxbe44}}

library(foreign) library(dplyr) library(ggplot2) library(readxl)

\hypertarget{uxb370uxc774uxd130-uxbd88uxb7ecuxc624uxae30-uxac00uxad6cuxc6a9-uxb370uxc774uxd130}{%
\section{데이터 불러오기 (가구용
데이터)}\label{uxb370uxc774uxd130-uxbd88uxb7ecuxc624uxae30-uxac00uxad6cuxc6a9-uxb370uxc774uxd130}}

f = read.spss(file = ``Koweps\_h15\_2020\_beta1.sav'',to.data.frame=T)

\hypertarget{uxb370uxc774uxd130-uxac80uxd1a0}{%
\section{데이터 검토}\label{uxb370uxc774uxd130-uxac80uxd1a0}}

head(f) View(f) dim(f) str(f) summary(f)

\hypertarget{uxbcc0uxc218uxba85-uxbc14uxafb8uxae30}{%
\section{변수명 바꾸기}\label{uxbcc0uxc218uxba85-uxbc14uxafb8uxae30}}

welfare = rename(f, region = h15\_reg7, gender = h1501\_4, age =
h1501\_5, working\_ability = h1503\_2, main\_working\_status = h1503\_4,
category\_business = h1503\_7 ) '\,'\,' region = 지역 gender = 성별 age
= 태어난 연도 working\_ability = 근로 능력 main\_working\_status = 주된
경제활동 참여상태 category\_business = 업종

관광데이터와 연관 가능 부동산 상권과 연관 가능 '\,'\,'

\hypertarget{uxd544uxc694uxd55c-uxbcc0uxc218uxb85cuxb9cc-uxc774uxb8e8uxc5b4uxc9c4-uxb370uxc774uxd130-uxd504uxb808uxc784-uxc0dduxc131}{%
\section{필요한 변수로만 이루어진 데이터 프레임
생성}\label{uxd544uxc694uxd55c-uxbcc0uxc218uxb85cuxb9cc-uxc774uxb8e8uxc5b4uxc9c4-uxb370uxc774uxd130-uxd504uxb808uxc784-uxc0dduxc131}}

welfare = data.frame(welfare\(region,  welfare\)gender,
welfare\(age,  welfare\)working\_ability,
welfare\(main_working_status,  welfare\)category\_business )

welfare = rename(welfare, region = welfare.region, gender =
welfare.gender, age = welfare.age, working\_ability =
welfare.working\_ability, parti\_status = welfare.main\_working\_status,
business = welfare.category\_business )

\hypertarget{uxb370uxc774uxd130-uxc7acuxac80uxd1a0}{%
\section{데이터 재검토}\label{uxb370uxc774uxd130-uxc7acuxac80uxd1a0}}

head(welfare) View(welfare) str(welfare) dim(welfare) summary(welfare)

\hypertarget{uxc804uxcc98uxb9ac}{%
\section{전처리}\label{uxc804uxcc98uxb9ac}}

\hypertarget{uxc9c0uxc5ed}{%
\section{지역}\label{uxc9c0uxc5ed}}

region =
c(`서울',`수도권(인천/경기)',`부산/경남/울산',`대구/경북',`대전/충남',`강원/충북',`광주/전남/전북/제주도')
for (i in 1:7)\{ welfare\(region[welfare\)region == i{]} = region{[}i{]}
\} \# 성별 welfare\(gender = ifelse(welfare\)gender == 1,`남',`여')

\hypertarget{uxb098uxc774}{%
\section{나이}\label{uxb098uxc774}}

welfare\(age = 2021-welfare\)age

\hypertarget{uxadfcuxb85c-uxb2a5uxb825-uxc815uxb3c4}{%
\section{근로 능력 정도}\label{uxadfcuxb85c-uxb2a5uxb825-uxc815uxb3c4}}

ability = c(`만 14세 이하',`근로 가능',`단순 근로 가능', `단순 근로
미약자',`근로 불가') for (i in 1:5)\{
welfare\(working_ability[welfare\)working\_ability == i-1{]} =
ability{[}i{]} \}

\hypertarget{uxc8fcuxb41c-uxacbduxc81cuxd65cuxb3d9-uxcc38uxc5ec-uxc0c1uxd0dc}{%
\section{주된 경제활동 참여
상태}\label{uxc8fcuxb41c-uxacbduxc81cuxd65cuxb3d9-uxcc38uxc5ec-uxc0c1uxd0dc}}

status = c(`상용직 임금근로자', `임시직 임금근로자', `일용직
임금근로자', `자활근로, 공공근로, 노인일자', `고용주', `자영업자',
`무급가족종사자', `실업자(지난 4주간 적극적으로 구직활동을 함)', `비경제
활동인구') for (i in 1:9)\{
welfare\(parti_status[welfare\)parti\_working\_status == i{]} =
status{[}i{]} \}

\hypertarget{uxc5c5uxc885-uxb300uxbd84uxb958uxb85c-uxc5c5uxc885uxc744-uxbcd1uxd569}{%
\section{업종 대분류로 업종을
병합}\label{uxc5c5uxc885-uxb300uxbd84uxb958uxb85c-uxc5c5uxc885uxc744-uxbcd1uxd569}}

business = c(`농업,임업 및 어업',`광업',`제조업',`전기,가스,증기 및 공기
조절 공급업', `수도,하수 및 폐기물 처리, 원료 재생업',`건설업',`도매 및
소매업',`운수 및 창고업', `숙박 및
음식점업',`정보통신업',`부동산업',`전문,과학 및 기술 서비스업',`사업시설
관리, 사업 지원 및 임대 서비스업', `공공 행정,국방 및 사회보장
행정',`교육 서비스업',`보건업 및 사회복지 서비스업', `예술,스포츠 및
여가관련 서비스업',`협회 및 단체, 수리 및 기타 개인 서비스업', `가구 내
고용활동 및 자가소비 생산활동',`국제 및 외국기관')

welfare\(business[welfare\)business \textless= 3{]} = business{[}1{]}
welfare\(business[welfare\)business \textless= 8{]} = business{[}2{]}
welfare\(business[welfare\)business \textless= 34{]} = business{[}3{]}
welfare\(business[welfare\)business \textless= 35{]} = business{[}4{]}
welfare\(business[welfare\)business \textless= 39{]} = business{[}5{]}
welfare\(business[welfare\)business \textless= 42{]} = business{[}6{]}
welfare\(business[welfare\)business \textless= 47{]} = business{[}7{]}
welfare\(business[welfare\)business \textless= 52{]} = business{[}8{]}
welfare\(business[welfare\)business \textless= 56{]} = business{[}9{]}
welfare\(business[welfare\)business \textless= 63{]} = business{[}10{]}
welfare\(business[welfare\)business \textless= 68{]} = business{[}11{]}
welfare\(business[welfare\)business \textless= 73{]} = business{[}12{]}
welfare\(business[welfare\)business \textless= 76{]} = business{[}13{]}
welfare\(business[welfare\)business \textless= 84{]} = business{[}14{]}
welfare\(business[welfare\)business \textless= 85{]} = business{[}15{]}
welfare\(business[welfare\)business \textless= 87{]} = business{[}16{]}
welfare\(business[welfare\)business \textless= 91{]} = business{[}17{]}
welfare\(business[welfare\)business \textless= 96{]} = business{[}18{]}
welfare\(business[welfare\)business \textless= 98{]} = business{[}19{]}
welfare\(business[welfare\)business \textless= 99{]} = business{[}20{]}

welfare = welfare \%\textgreater\% filter(!working\_ability == `근로
불가') welfare = welfare \%\textgreater\% filter(!is.na(business))
welfare = welfare \%\textgreater\% filter(!parti\_status == `비경제
활동인구')

\hypertarget{uxc804uxcc98uxb9ac-uxc644uxb8cc}{%
\section{전처리 완료}\label{uxc804uxcc98uxb9ac-uxc644uxb8cc}}

table(welfare\(region) table(welfare\)gender)
table(welfare\(age) table(welfare\)working\_ability)
table(welfare\(parti_status) table(welfare\)business)

Seoul = welfare{[}welfare\$region == `서울',{]} head(Seoul) ggplot(data
= Seoul, aes(x = age,y = business)) + geom\_point()+ggtitle(``서울의
연령별 업종 분포'')+ theme(plot.title = element\_text(family =
``serif'', face = ``bold'', hjust = 0.5, size = 15, color =
``darkblue''))

Sudogwun = welfare{[}welfare\$region == `수도권(인천/경기)',{]}
head(Sudogwun) ggplot(data = Sudogwun, aes(x = age,y = business)) +
geom\_point()+ggtitle(``수도권(인천/경기)의 연령별 업종 분포'')+
theme(plot.title = element\_text(family = ``serif'', face = ``bold'',
hjust = 0.5, size = 15, color = ``darkblue''))

Junra\_Jeju = welfare{[}welfare\$region == `광주/전남/전북/제주도',{]}
head(Junra\_Jeju) ggplot(data = Junra\_Jeju, aes(x = age,y = business))
+ geom\_point()+ggtitle(``광주/전남/전북/제주도의 연령별 업종 분포'')+
theme(plot.title = element\_text(family = ``serif'', face = ``bold'',
hjust = 0.5, size = 15, color = ``darkblue''))

table(Seoul\(age,Seoul\)business)

\end{document}
